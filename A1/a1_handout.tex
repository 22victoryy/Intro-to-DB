\documentclass[boldsans]{article}
\usepackage{ccfonts,fullpage,graphics,rotating,amsmath,amssymb}
\usepackage{hyperref}
\hypersetup{pdfpagemode=Fullscreen,
  colorlinks=true,
  linkfileprefix={}}
\title{csc343 winter 2020\\assignment \#1: relational algebra\\
  {\bf due February 7th, 4 p.m.}}
\renewcommand{\today}{~}
\newcommand{\ul}{$\underline{~}$}
\newboolean{solutions}
\setboolean{solutions}{false}
\begin{document}
\maketitle\vspace{-2\baselineskip}
\noindent\section*{goals}
This assignment aims to help you learn to:
\begin{itemize}
\item read a relational scheme and analyze instances of the schema
\item read and apply integrity constraints
\item express queries and integrity constraints of your own
\item think about the limits of what can be expressed in relational algebra
\end{itemize}
Your assignment must be typed to produce a PDF document
\textbf{a1.pdf} (hand-written submissions are not acceptable).  You
may work on the assignment in groups of 1 or 2, and submit a
single assignment for the entire group on
\href{https://markus.cdf.toronto.edu/csc343-2020-01}{MarkUs}.  You
must establish your group well before the due date by submitting an
incomplete, or even empty, submission.
\section*{background}
You will be working on a schema and queries for a database used by a
zoological institute to track an archive of their artifacts.

During a field trip collectors gather a variety of artifacts of the
animals they study, resulting in tissue samples, images, physical
models (such as casts of paw prints), or live colonies.

After arriving at the institute, artifacts must be safely stored and
maintained by technicians.  Some artifacts are cited in one or more
publications.  In all cases the official species name must be
recorded, and must appear in the
\href{https://www.catalogueoflife.org/}{Catalogue of Life database}.
If correct taxonomic practices are followed, each species belongs to
exactly one genus, and each genus to exactly one family. Tables COL,
Genus, and Species are derived from Catalogue of Life database.

\section*{relations}
\begin{itemize}
\item Collection(\underline{CID}, date, SID)\\
  Tuples here represent entire collections from a field trip, where
  \textit{CID} is the collection ID, \textit{date} is the starting
  date of the field trip, and \textit{SID} is the staff ID of the
  collector.

\item Collected(\underline{CID, AN})\\
  A tuple here represents the fact that collection \textit{CID}
  includes artifact number \textit{AN}.  A single collection usually
  contains multiple artifacts, and a single artifact may be aggregated
  from more than one collection.

\item Artifact(\underline{AN}, species, type, location, SID)\\
  Tuples here represent single artifact collected in the field.
  \textit{AN} is the artifact number, \textit{species} is the scientific species
  name, \textit{type} is one of tissue, image, model, or live,
  \textit{location} is where it was collected, and \textit{SID} is the
  staff number of the technician who maintains this artifact.

\item Published(\underline{AN, journal, date})\\
  A tuple here represents the fact that artifact \textit{AN} was
  mentioned in scholarly publication \textit{journal} with publication
  date \textit{date}.

\item Staff(\underline{SID}, name, email, rank, date)\\
  These tuples represent a member of the institute's scientific staff.
  \textit{SID} is the staff ID, \textit{name} is their full name,
  \textit{email} is their professional email, \textit{rank} is one of:
  technician, student, pre-tenure, or tenured, and
  \textit{date} is the date when they attained that rank.

\item COL(\underline{family})\\
  A singleton tuple here means that \textit{family} is a scientific
  zoological family name that appears in the Catalogue of  Life.

\item Genus(\underline{genus}, family)\\
  A tuple here means that \textit{genus} is in family \textit{family}.

\item Species(\underline{species}, genus)\\
  A tuple here means that \textit{species} is in genus \textit{genus}.
\end{itemize}

\section*{our constraints}

For each of the following constraints give a one sentence explanation of what the
constraint implies, and why it is required.

\begin{itemize}
\item $\pi_{species}(Artifact) - \pi_{species}(Species) = \emptyset$.
\item $\pi_{rank}(Staff) \subseteq \{\text{'technician', 'student',
    'pre-tenure', 'tenure'}\}$.
\item $\pi_{family}(Genus) - \pi_{family}(COL) = \emptyset$.
\item $\pi_{genus}(Species) \subseteq \pi_{genus}(Genus)$.
\item $\pi_{CID}(Collected) = \pi_{CID} (Collection)$.
\item $\pi_{AN}(Artifact) = \pi_{AN}(Collected)$.
\item $\pi_{SID}(Collection) \subseteq \pi_{SID}(Staff)$.
\item $\pi_{SID}(Artifact) \subseteq \pi_{SID}(Staff)$.
\item $\pi_{type}(Artifact) \subseteq \{'tissue', 'image', 'model', 'live'\}$
\item $\pi_{AN}(Published) \subseteq \pi_{AN}(Artifact)$
\end{itemize}

\section*{queries}

Write relational algebra expressions for each of the queries below.
You must use notations from this course and operators:
\begin{equation*}
  \pi, \sigma, \rho, \bowtie, \bowtie_{condition}, \times, \cap, \cup,
  -, =
\end{equation*}
You may also use constants:
\begin{equation*}
  \text{today (for current date)}\qquad\emptyset\text{ (for the empty set)}
\end{equation*}
In your queries pay attention to the following:
\begin{itemize}
\item All relations are sets, and you may only use relational algebra
  operators covered in Chapter~2 of the course text.

\item Do not make assumptions that are not enforced by our constraints
  above, so your queries should work correctly for any database that
  obeys our schema and constraints.

\item Other than constants such as 23 or "lupus", a select operation
  only examines values contained in a tuple, not aggregated over an
  entire column.

\item Your selection conditions can use arithmetic operators, such as
  $+, \leq, \neq, \geq, >, <$ and friends.  You can use logical
  operators such as $\vee, \wedge$, and $\neg$, and treat dates and
  numeric attributes as numbers that you can perform arithmetic on.

\item Use good variable names and provide lots of comments to explain
  your intentions.

\item Return multiple tuples if that is appropriate for your query.
\end{itemize}

There may be a query or queries that cannot be expressed in the
relational algebra you have been taught so far, in which case just
write ``cannot be expressed.''  The queries below are not in any
particular order.

\begin{enumerate}
\item Rationale: Performance reviews include seeing how current the work
  is of staff who have held their current rank for a long time.

  \textbf{Query:} Find the most recent collection date of any artifact
  collected by a staff member who has held their current rank the
  longest.  Keep ties.
  
\item Rationale: Staff who maintain every artifact in some collection
  should be considered favourably in performance reviews.

  \textbf{Query:} Find all staff who maintain all artifacts in at
  least one collection.

\item Rationale: An artifact collected and maintained by the same staff
  may have some special requirements that should be investigated.

  \textbf{Query:} Find all artifacts that were collected by the same
  staff who maintains them.

\item Rationale: Identify multi-talented field workers.

  \textbf{Query:} Find all staff who have collected at least 3
  artifacts from every species in some family.

\item Rationale: Which publications might have some specialized niche
  focus?

  \textbf{Query:} Find all publications that have used exactly 2 of
  our artifacts.

\item Rationale: Identify motherlode locations.

  \textbf{Query:} Find all locations where at least one artifact from
  every family has been collected.

\item Rationale: Exclusively tissue sample collectors may need extra
  support for special reagents and shipping costs.

  \textbf{Query:} Find all staff who have collected only tissue
  samples.

\item Rationale: Collection staff who should be encouraged to
  diversify their network.

  \textbf{Query:} Find all staff pairs who have worked only with each
  other on collections.

\item Rationale: Track the influence of a given staff member.

  \textbf{Query:} Staff member SID$_1$ is influenced by staff member
  SID$_2$ if (a) they have ever worked together on a collection or (b)
  if SID$_1$ has ever worked with a staff member who is influenced by
  SID$_2$.  Find SIDs of staff members influenced by SID 42.
\end{enumerate}

\section*{your constraints}

For each of these constraints you should derive a relational algebra
expression of the form $R = \emptyset$, where $R$ may be derived in
several steps, by assigning intermediate results to a variable.  If
the constraint cannot be expressed in the relational algebra you have
been taught, write ``cannot be expressed.''

\begin{enumerate}
\item No species is also a genus.
\item No genus belongs to more than one family.
\item All publications must be published after all artifacts they use
  have been collected.
\item Students may not catalogue live artifacts.
\end{enumerate}
\section*{submissions}

Submit \textbf{a1.pdf} on \href{https://markus.teach.cs.toronto.edu/csc343-2020-01}{MarkUs}.
One submission per group, whether a group is one or two people.  You
declare a group by submitting an empty, or partial, file, and this
should be done well before the due date.  You may always replace such
a file with a better version, until the due date.

Double check that you have submitted the correct version of your file
by downloading it from MarkUs.

\section*{marking}

We mark your submission for correctness, but also for good form:
\begin{itemize}
\item For full marks you should add comments to describe the \textit{data}, rather than
  \textit{technique}, of  your queries.  These may help you get part
  marks if there is a flaw in your query.
\item Please use the assignment operator, ``:='' for intermediate results.
\item Name relations and attributes in a manner that helps the reader
  remember their intended meaning.
\item Format the algebraic expressions with line breaks and formatting
  that help make the meaning clear.
\end{itemize}
\end{document}
